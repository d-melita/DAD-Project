
%
%  $Description: DADTKV report$ 
%
%  $Author: Diogo Melita, Martim Moniz e Mateus Pinho$
%  $Date: 27/10/2023$
%  $Revision:$
%

\documentclass[times, 10pt,twocolumn]{article} 
\usepackage{latex8}
\usepackage{times}
\usepackage{algorithm}
\usepackage[noend]{algpseudocode}
\usepackage{enumitem}

%\documentstyle[times,art10,twocolumn,latex8]{article}

%------------------------------------------------------------------------- 
% take the % away on next line to produce the final camera-ready version 
\pagestyle{empty}

%------------------------------------------------------------------------- 
\begin{document}

\title{DADTKV}

\author{Diogo Melita, Martim Moniz, Mateus Pinho\\
Universidade de Lisboa - Instituto Superior Técnico\\
MSc Compiter Science and Engineering
}

\maketitle
\thispagestyle{empty}

\begin{abstract}
       In the field of distributed systems, ensuring concurrent access to 
       shared data while maintaining consistency is a very challenging task. This 
       paper presents the design and implementation of DADTKV, a distributed 
       transactional key-value store. DADTKV uses a three-tier architecture 
       comprising client applications, transaction managers and lease managers. 
       Clients submit transactions, composed of read and write operations, to 
       transaction managers, which in turn acquire leases from lease managers. 
       The Paxos algorithm orchestrates lease assignments, regulating access to 
       shared data and ensuring both consistency and fault tolerance. Implemented 
       in C\# and using gRPC for remote communication, DADTKV demonstrates robustness 
       and efficiency in managing concurrent transactions across distributed environments.
       This project is an educational exploration that delves into distributed systems, 
       transaction management, and fault tolerance mechanisms. While not meant for 
       real-world use, DADTKV illustrates the practical application of theoretical 
       concepts, deepening our understanding of distributed systems principles in 
       a hands-on context.
\end{abstract}



%------------------------------------------------------------------------- 
\Section{Introduction}

In the realm of modern computing, managing shared data within distributed 
systems is a big challenge. With organizations relying on distributed 
architectures for critical applications, the need to ensure concurrent 
access, consistency, and fault tolerance has become paramount. With this in 
mind we developed DADTKV, a distributed transactional key-value store to 
manage data objects (each one being a \textless key, value\textgreater\ pair) 
that reside in server memory and can be accessed concurrently by programs that 
execute in different machines.

\SubSection{Architecture}
In the DADTKV system, the architectural framework consists of three tiers. 
The initial tier comprises client applications, the users of the DADTKV system, 
responsible for executing transactions on the stored data. These transactions, 
which are composed of read and write operations, are then routed to the second 
tier, comprised of transaction managers. These managers not only provide transactional 
access to the data but also ensure the replication of transaction updates for data 
durability. To manage concurrent access effectively, a system of leases is employed. 
These leases, determining access rights, are acquired by transaction managers from 
lease manager servers, the third tier. The assignment of leases to specific 
transaction managers is accomplished through the Paxos algorithm executed among 
the lease manager servers.

%------------------------------------------------------------------------- 
\Section{Data Model and Transactions}

\SubSection{Data}
Applications utilizing DADTKV exclusively interact with a specific object type 
called DadInt. Each DadInt object operates as a key-value pair, with the key 
represented as a string and the corresponding value as an integer.\\

\SubSection{Transactional Operations}
In DADTKV, transactions involve reading and modifying DadInt objects. Clients 
submit transactions specifying the DadInt objects they want to access and 
modify. Read operations retrieve current values, while write operations 
modify selected DadInt object values. These operations enable complex tasks, 
ensuring shared data integrity.

\SubSection{Ensuring Strict Serializability}
DADTKV ensures strict serializability. Transactions first execute 
read operations and then write operations sequentially. This ordered processing 
prevents conflicts, maintaining consistent shared data. Adhering to strict 
serializability guarantees DADTKV's reliability, making it a robust solution for 
concurrent data access.\\

%------------------------------------------------------------------------- 
\Section{Transactional Operations in DADTKV}

In the context of DADTKV, transactional operations are at the core of managing 
shared data and ensuring consistency across distributed environments. The system 
provides two fundamental methods, TxSubmit and Status, within its client library 
to facilitate transaction processing and monitor transaction states. In our tests, the sequence 
of operations executed by each client is predefined by the script which is executed 
sequentially and in loop until the program is closed. The script also includes a 
"wait" instruction (for a number of milliseconds) in order to insert a waiting time 
between other commands.

\SubSection{TxSubmit}
This method initiates the execution of a transaction within a transaction manager. 
It takes three parameters: a string identifying the client initiating the transaction, 
a list of strings specifying the DadInt objects to be read, and a list of DadInt 
objects to be modified by the transaction. The function returns a list of DadInt 
objects, representing the results of the transaction's read operations. 
In case of a transaction abortion, the return value comprises a single DadInt 
object with the key "abort". The client will try to send the transaction request to 
the first transaction manager that it is assigned to, i.e., client 1 will first 
try to communicate with TM1, and if it is unable to do so, it will try with the next 
transaction manager and so on.

\begin{algorithm}
    \caption{Begin execution of transaction}
    \begin{algorithmic}[1]
        \Function{TxSubmit}{id, reads, writes}
            \State $request \gets newRequest(id)$
            \State $request.add(reads)$
            \State $request.add(writes)$
            \State $tm \gets getTransactionManager$
            \While{transaction not sent}
                \State $response \gets tm.Send(request)$
                \If{response $\not=$ null}
                    \Return $DadInts Read$
                \Else{}
                    \State $tm \gets get\ next\ Transactio\ Manager$
                \EndIf
            \EndWhile
        \EndFunction
    \end{algorithmic}
\end{algorithm}

\SubSection{Status}
This method is used by the clients to request the status of every 
node in the system. The client will send a status request to every 
node of the system. When a node receives the Status Request, 
it will display on its console its state.

\begin{algorithm}
    \caption{Status Request}
    \begin{algorithmic}[1]
        \Function{Status}{}
            \State $request \gets newStatusRequest()$
            \ForAll{Nodes}
                \State send(request)
            \EndFor
        \EndFunction
    \end{algorithmic}
\end{algorithm}

\Section{Leases}
Transaction managers employ leases to secure exclusive access 
to a specific set of DadInts. Once acquired, a lease remains in 
the possession of the requesting transaction manager indefinitely, 
until it identifies a conflicting lease allocated to another 
transaction manager. Leases enable a transaction manager to 
execute multiple transactions that necessitate the same lease 
without the need for consensus. However, this advantage is only 
significant if transactions managed by different entities are 
expected to access distinct data sets, reducing the frequency of 
lease acquisitions.

\Section{Transaction Managers}

Transaction managers in the DADTKV are a set of nodes of the 
system that play a pivotal role in ensuring the integrity and 
consistency of shared data across distributed environments. 
Responsible for processing and coordinating transactions initiated 
by clients, these managers orchestrate a series of operations to 
guarantee strict serializability and fault tolerance. All 
transaction managers store a full copy of the set of DadInt in 
the system.

\SubSection{Transaction Execution Workflow}
Upon receiving a transaction request from the client, the 
transaction manager executes a predefined workflow in order 
to process the transaction request.

\begin{enumerate}[label=\textbf{Step \arabic*:}, left=0pt, align=left, itemsep=0.5em]
    \item[Lease Checking:]
    At the moment of receiving a transaction request, the 
    transaction manager will first check if it holds the appropriate 
    leases for the DadInts included in the request. If it does, 
    it will jump to the Transaction Processing step, otherwise it 
    will need to request the leases from Lease Managers.
    
    \item[Acquiring Leases:]
    If a transaction Manager does not have the appropriate leases 
    to execute the transaction received, it will send a Lease 
    Request to the Lease Managers. However, the response from 
    Lease Managers will arrive asynchronously, at the beginning 
    of the next slot.
    
    \item[Transaction Processing:]
    Once a transaction manager has obtained all the necessary 
    leases for the received transaction, it proceeds with its 
    execution. Initially, it processes the read operations to 
    prevent conflicts, ensuring the consistency of shared data. 
    Subsequently, the transaction manager executes the write operations, 
    as detailed in the following explanation.
\end{enumerate}

\SubSection{Processing Write Operations}
Following the completion of read operations, Transaction Managers 
proceed to handle write operations. Immediate processing isn't 
feasible due to the necessity of propagating these changes to 
other Transaction Managers to maintain stable and consistent data. 
To ensure data integrity, we've implemented a robust approach - 
the Two-Phase-Commit protocol, augmented with a majority consensus.

In the initial phase, the initiating Transaction Manager 
dispatches a Prepare message to other Transaction Managers, 
awaiting their responses. The transaction can only proceed if 
it receives affirmative responses from a majority. Should this 
condition not be met, the transaction is aborted and the client 
will receive a single DadInt with the key "abort".

Upon a successful preparation phase, the transaction enters the 
Commit Step. Here, the initiating Transaction Manager communicates 
the write operations to be executed by other Transaction Managers. 
These managers promptly execute the transactions, and the 
originating Transaction Manager processes them subsequently.

This methodology, employing the Two-Phase-Commit protocol coupled 
with a majority consensus, ensures that transactions are executed 
only when a majority of Transaction Managers are available. Even if 
a subset of managers does not receive the Commit request, at 
least one Transaction Manager will execute the transaction. 
Additionally, by maintaining a comprehensive write log of operations, 
the system conducts a synchronization check at the onset of each slot, 
ensuring seamless coordination across all Transaction Managers.

\begin{algorithm}
    \caption{Transaction Processing}
    \begin{algorithmic}[1]
        \Function{ExecuteTransaction}{transaction}
            \ForAll{Transaction Managers}
                \State send(PrepareRequest)
            \EndFor
            \If{\#PrepareResponses \textgreater\ \#TMs/2 + 1}
                \State send(CommitRequest(transaction.Writes))
                \State processWriteOperations(transaction.Writes)
            \EndIf
        \EndFunction
    \end{algorithmic}
\end{algorithm}

\SubSection{Slot Preparation}
At the beginning of each time slot, every Transaction Manager 
undergoes a series of updates and checks to ensure system integrity. 
Initially, the manager updates its status to either "crashed" or 
"normal" based on the configuration file and adjusts its list of 
crashed processes accordingly.

Subsequently, the Transaction Manager attempts to update its 
transaction log status. During this phase, it requests written 
operations logs from other managers. At least one manager is 
guaranteed to have an updated log which will be the largest one, 
and in this case, the requesting manager synchronizes with the 
one possessing the largest log. If the manager's log is outdated, 
it relinquishes all held leases and aligns its local log with 
the most extensive and common log received.

Finally, the Transaction Manager contacts the Lease Managers for 
a Status Update to obtain newly assigned leases. After acquiring 
these leases, the manager checks for conflicting leases and 
reviews pending transactions. If the manager possesses the necessary 
leases to execute a transaction and there are no conflicting leases 
within the same time slot, the transaction is processed as mentioned 
above.

\begin{algorithm}
    \caption{Slot Preparation}
    \begin{algorithmic}[1]
        \Function{PrepareSlot}{}
            \State $Tm.state \gets Config.GetState(TM)[slotID]$
            \ForAll{TransactionManagers}
                \If{TM.state() == crashed}
                    \State crashedProcesses.Add(TM)
                \EndIf
            \EndFor
            \State UpdateLog()
            \State AskLeaseManagersForUpdate()
        \EndFunction
        \Function{UpdateLog}{}
            \State $logs \gets new List()$
            \State $request \gets new UpdateLogRequest()$
            \ForAll{not crashed and not suspected TMs}
                \State $response \gets TM.sendLog(request)$
                \State logs.Add(response)
            \EndFor
            \State GetMajorityResponses()
            \State $largestLogSize \gets LargestCommunLogSize()$
            \If{currentLog.size() \textless\ largestLogSize}
                \State UpdateCurrentLog()
            \EndIf
        \EndFunction
        \Function{AskLeaseManagersForUpdate}{}
            \State $request \gets new UpdateStatusRequest()$
            \ForAll{LeaseManagers}
                \State sendUpdateStatusRequest(request)
            \EndFor
            \State CheckConflictingLeases(UpdateStatusResponse)
            \ForAll{Pending Transactions}
                \If{All leases and No Lease Conflicts}
                    \State ExecuteTransaction(transaction)
                \EndIf
            \EndFor
        \EndFunction
    \end{algorithmic}
\end{algorithm}

\SubSection{Implementation Decisions}
In our implementation, we opted for a modified Two-Phase Commit protocol over a single-phase commit for several compelling reasons. Firstly, our system 
demands strong consistency and transactional guarantees, even if 
it means accommodating potential blocking and heightened latency 
associated with synchronous coordination. Two-Phase Commit ensures 
the atomicity of transactions, a critical aspect that either 
commits all Transaction Managers' transactions or none at all. 
This property is paramount to maintaining the system's correctness 
and preventing inconsistencies.

To balance the need for consistency with client expectations of 
low latencies and high availability, we devised a version
of the protocol using majorities. When our processes communicate 
with clients, they wait for a majority of "Ok" responses from Prepare 
Requests before proceeding with transactions. This method grants our
managers the confidence to proceed, as a majority commitment ensures 
that the transactions will be executed.

Moreover, at the beginning of each new slot, we conduct a preparation 
phase to guarantee that every manager in DADTKV is up to date. This 
step is crucial because failures might occur during the execution 
of the Two-Phase Commit protocol, potentially causing a manager 
to miss the commit message containing the corresponding write 
operations. By performing this preparation, we ensure consistency 
throughout our system, verifying that every manager is impeccably updated.



\Section{Lease Managers}
The role of lease managers is to order Lease Requests and assign 
leases to transaction managers, using the Paxos algorithm. 

\SubSection{Slot Preparation}
At the start of every time slot, each Lease Manager updates its 
status based on the configuration file, for the other Lease Managers 
checks which are crashed and stores that information and, finally, 
it initiates a new Paxos epoch. However, only the Leader has the 
authority to commence the new Paxos epoch by sending a "prepare"
message to the other Lease Managers, who await this signal.

\SubSection{Paxos Implementation}
A Paxos epoch commences at the start of a new slot and is triggered 
by the Leader, only if no existing Paxos epoch is running and no 
value has been decided yet. The Leader Lease Manager initiates 
the process by sending a prepare request to the appropriate Lease 
Managers and awaits a majority of promise replies. Lease Managers 
responding to the prepare request include their read timestamp and 
leased information.

Once a majority of responses is gathered, the leader examines all 
received replies to identify the most recent one, determined by the 
read timestamp. If a more recent Lease Manager is found, it assumes 
the role of the leader. Subsequently, the Paxos slot progresses to 
the next step, where the leader proposes lease values. It sends an 
accept request to all appropriate Lease Managers and awaits the decision 
made by learners.

Upon receiving an accept request, Lease Managers reply with their 
write timestamp and leases in a decide request. Learners then decide 
on a value by selecting the most common response received and relay 
this information. Upon completion of a Paxos slot, each Lease Manager 
stores the decided leases and provides them to Transaction Managers 
upon request.

\begin{algorithm}
\caption{Paxos}
    \begin{algorithmic}
        \Function{Paxos}{}
            \If{Paxos Leader}
                \State sendPrepareResquest()
            \EndIf
            \ForAll{PromiseResponses}
                \If{response.ReadTimeStamp \textgreater\ LeaderID}
                    \State WaitPaxos()
                \EndIf
            \EndFor
            \State Leader.SendAccept(value)
            \State WaitForLearnersToDecide()
        \EndFunction
    \end{algorithmic}
\end{algorithm}

\Section{Conclusion}

In conclusion, our endeavor to design and implement the DADTKV 
system has been a significant exploration into the complexities 
of distributed systems, transaction management, and fault tolerance 
mechanisms. Through meticulous planning and rigorous implementation, 
we have achieved the successful realization of a distributed transactional 
key-value store that upholds strict serializability, consistency, and 
fault tolerance.

Our work demonstrates the vital role of transaction managers and 
lease managers in orchestrating concurrent access to shared data 
while ensuring data integrity. By opting for the Two-Phase Commit 
protocol, enhanced with majorities, over gossip-based solutions, 
we have prioritized strong consistency, transactional guarantees, 
and system stability. The use of leases, managed through the Paxos 
algorithm, has enabled us to regulate access to shared data effectively, 
ensuring non-conflicting transactions can proceed concurrently.

Throughout this project, we have encountered and overcome various 
challenges, leading to valuable insights into the nuances of 
distributed systems. Our implementation decisions, guided by theoretical 
concepts, have translated into a robust and efficient system capable 
of managing concurrent transactions across distributed environments.

\nocite{ex1, ex2}
\bibliographystyle{latex8}
\bibliography{latex8}

\end{document}
